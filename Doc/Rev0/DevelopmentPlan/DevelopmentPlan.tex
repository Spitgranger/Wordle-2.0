\documentclass{article}
\usepackage[utf8]{inputenc}
\usepackage[margin=1in]{geometry}
\usepackage[titletoc,title]{appendix}
\usepackage{hyperref}
\usepackage{booktabs}
\usepackage{tabularx}
\usepackage{enumitem}

\title{SE 3XA3: Development Plan\\Wordle 2.0}

\author{Group 8
	\\ Richard Fan, fanr13
	\\ Noel Zacharia, zacharin
	\\ Biranugan Pirabaharan, pirabahb
}

\date{}

\begin{document}
	
	\begin{table}[hp]
		\caption{Revision History} \label{TblRevisionHistory}
		\begin{tabularx}{\textwidth}{llX}
			\toprule
			\textbf{Date} & \textbf{Developer(s)} & \textbf{Change}\\
			\midrule
			2022/02/01 & Noel Zacharia & Initial draft, Sections 1,7,8\\
			2022/02/03 & Richard Fan & Team Meeting Plan, Team Communication 
			Plan, Team Roles\\
			2022/02/03 & Biranugan Pirabaharan & Sections 5 and 6\\
			2022/02/04 & Richard Fan & Corrected grammar and word choice\\
			\bottomrule
		\end{tabularx}
	\end{table}
	
	\newpage
	
	\maketitle
	
	\section{Introduction}
	This document is the Development Plan for the project 
	\href{https://gitlab.cas.mcmaster.ca/zacharin/wordle_clone_3xa3_l01_group8}{Wordle
		2.0}.
	
	\section{Team Meeting Plan}
	Below is the team meeting schedule.
	\begin{table}[hp]
		\begin{tabularx}{\textwidth}{|l|l|X|}
			\toprule
			\textbf{Where} & \textbf{When} & \textbf{Length}\\
			\midrule
			MS Teams L01 & Every Tuesday and Thursday @ 9:30 AM & 2 Hours Each 
			Session\\
			Facebook Messenger & Every Saturday at 9:00 PM & 2 Hours\\
			\bottomrule
		\end{tabularx}
		\caption{Team Meeting Plan} \label{tab:teamMeetingPlan}
	\end{table}
	\newline
	The team has decided to meet during the two weekly lab sessions and 
	Saturday. Additional sessions may be added if deemed necessary by the team.
	\subsection{Agenda Rules}
	\begin{enumerate}
		\item A meeting agenda will be completed before every meeting 
		containing 
		the 
		following information:
		\begin{enumerate}
			\item Outline of the meeting
			\item Meeting topics/issues
			\item Who is to lead each topic
			\item Approximate timeline for each topic
		\end{enumerate}
		\item All team members will contribute to the agenda in between 
		meetings.
		\item The agenda is to be uploaded to the repository 10 minutes before 
		the start of the meeting.
		\item The chair will be responsible for the smooth flow of the meeting.
		\item A member will be assigned to record the minutes of the meeting.
		\item All members of the team shall be present unless otherwise 	
		communicated.
		\item If a member is unable to attend a meeting, they must let another 
		member know so the meeting can be rescheduled accordingly.
	\end{enumerate}
	\subsection{Meeting Roles}
	The following meeting roles have been defined:
	\begin{description}
		\item[Chair] The role of the chair is to ensure that the meeting 
		progresses 
		smoothly. They lead the meeting according to the set agenda. 
		They 
		let members know when it is their turn to speak 
		and when their turn ends. They may also be responsible for facilitating 
		conflict resolution.
		\item[Recorder] The role of the recorder is to keep a record of the 
		meetings. This may include key decisions, problems, and solutions 
		discussed. The recorder is responsible for compiling this record into a 
		standard document.
		\item[Participant] The role of the participant is to actively attend 
		and 
		contribute to the meeting. They should participate in any discussion 
		and be 
		prepared to offer their thoughts on topics presented. 
	\end{description}
	
	\section{Team Communication Plan}
	Meetings are to be conducted over voice calls through Microsoft Teams or 
	Facebook Messenger. Team members should be present at the scheduled start 
	time. 
	A 
	Facebook group chat has also been created for team members. This group chat 
	will be used for any urgent communication outside of scheduled meetings. 
	Any 
	issues with the code shall be flagged and communicated using Git issues.
	
	\section{Team Member Roles}
	It is decided that the roles will rotate each meeting. Each member of 
	the 
	team will take turns becoming the Chair, Recorder, and Participant. The 
	roles 
	will rotate based on each member's last name to ensure that members do not 
	repeat roles in the next meeting. Since team members have similar 
	experiences 
	in the technology used, all team members have agreed to work on 
	all 
	aspects of the project. The assignment of work will be decided and agreed 
	upon by all team members. This assignment will be documented using a Gantt 
	chart.
	
	\section{Git Workflow Plan}
	As Dr.Bokhari suggested in lecture we will be using a feature-branch 
	workflow model. With this model, we will develop all the project’s 
	features in separate branches. This way developers can update and peer
	review the features without concern of other work being affected. 
	Eventually, these branches are to be merged to the main branch after the
	features have been tested and the developers are fully confident in its 
	functionality. Thus, throughout the term and the lifetime of the project,
	the main branch will remain fully functional. All commits must include
	clear, descriptive messages. This helps with any issue tracking and allows
	for easy resets or reverts to previous stable versions if they are needed. 
	We will use milestones that align with and follow the deliverables as they 
	are listed in the outline. This is to help stay on track and meet deadlines.
	
	\section{Proof of Concept Demonstration Plan}
	We have limited experience when it comes to testing JavaScript. We will
	need to explore our options for testing to ensure we can develop the best
	tests we can. Currently, we are learning more about frameworks such as JEST
	and Cypress. We hope to have a better understanding before developing the
	test plan. This will help us understand the scope of the tests we can
	develop. We are not very familiar with React and TypeScript. We will have a
	bit of a learning curve to grasp the basics, so we can better understand
	the source code. This will require us to spend some time before development
	to practice and explore the nuances of the languages used in the original
	code. Through the plethora of online resources, we should be able to learn
	everything we need. To help get through these hurdles, we will attempt to
	create a minimalistic high-fidelity prototype of the game to ensure we can
	learn the basics for when we start the full development process. This
	version will not be flawless but simply a first attempt at converting the
	source code into our implementation. This simply serves to confirm we
	have the knowledge to do so and gives us good practice.
	
	\section{Technology}
	\subsection{Programming Language}
	The code will be written using the latest version of JavaScript. JavaScript 
	was 
	chosen because it is a common programming language used in front-end 
	development. Team members have familiarity with JavaScript.
	\subsection{IDE}
	The IDE that will be used is Visual Studio Code. This is a relatively 
	common 
	IDE with great support for extensions.
	\subsection{Testing}
	DOM Testing will be done to ensure the components are rendered correctly. 
	We 
	will use a common testing framework such as JEST or Cypress to ensure unit 
	testing is done.
	\subsection{Documentation}
	This project will be documented using LaTeX and Doxygen and will be made 
	available in the project repository on GitLab.
	
	\section{Coding Style}
	The goal is to maintain readability. To ensure this we will be 
	attempting to consistently use PascalCase in our codebase.
	
	\section{Project Schedule}
	
	Our 
	\href{https://gitlab.cas.mcmaster.ca/zacharin/wordle_clone_3xa3_l01_group8/-/tree/main/ProjectSchedule}{Gantt
		chart} will be updated as the project progresses.
	
	\section{Project Review}
	Not applicable yet.
	
\end{document}
