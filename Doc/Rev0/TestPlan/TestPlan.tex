\documentclass[12pt, titlepage]{article}

\usepackage{booktabs}
\usepackage{tabularx}
\usepackage{hyperref}
\usepackage{lscape}
\usepackage{float}
\usepackage{multirow}

\hypersetup{
    colorlinks,
    citecolor=black,
    filecolor=black,
    linkcolor=red,
    urlcolor=blue
}
\usepackage[round]{natbib}

\title{SE 3XA3: Test Plan\\Wordle 2.0}

\author{Team 8, Goufs
		\\ Richard Fan, fanr13
		\\ Noel Zacharia, zacharin
		\\ Biranugan Pirabaharan, pirabahb
}

\date{\today}


\begin{document}

\maketitle

\pagenumbering{roman}
\tableofcontents
\listoftables
\listoffigures

\begin{table}[bp]
\caption{\bf Revision History}
\begin{tabularx}{\textwidth}{p{3cm}p{2cm}X}
\toprule {\bf Date} & {\bf Version} & {\bf Notes}\\
\midrule
Mar 1, 2022 & 1.0 & Initial Document\\
Mar 3, 2022 & 1.1 & Finished section 1 and 2\\
Mar 5, 2022 & 1.2 & Tests for functional requirements\\
Mar 9, 2022 & 1.3 & Tests for non functional requirements\\
Mar 10, 2022 & 1.4 & Traceability Matrices added\\
Mar 11, 2022 & 1.5 & Fixed grammar mistakes\\
\bottomrule
\end{tabularx}
\end{table}

\newpage

\pagenumbering{arabic}


\section{General Information}

\subsection{Purpose}
This document describes the tests, methods and tools that will be used to 
verify the functional and non-functional requirements for Wordle 2.0. These 
methods serve as a guide to test the software once it has been fully 
implemented. The purpose of these tests is to discover and correct any 
potential 
errors that may have occurred during implementation and ensure adherence to 
the Software Requirements Specification.

\subsection{Scope}
The test cases described within this document cover all requirements, both 
non-functional and functional, that were specified within the Software 
Requirements Specification. Tests will be conducted as development progresses 
and this document will be updated accordingly. 

\subsection{Acronyms, Abbreviations, and Symbols}

\begin{table}[hbp]
	\caption{\textbf{Table of Abbreviations}} \label{Table}
	
	\begin{tabularx}{\textwidth}{p{3cm}X}
		\toprule
		\textbf{Abbreviation} & \textbf{Definition} \\
		\midrule
		CSS & Cascading Style Sheets\\
		GUI & Graphical User Interface\\
		HTML & Hypertext Markup Language\\
		JS & Javascript\\
		\bottomrule
	\end{tabularx}
	
\end{table}

\begin{table}[!htbp]
	\caption{\textbf{Table of Definitions}} \label{Table}
	
	\begin{tabularx}{\textwidth}{p{3cm}X}
		\toprule
		\textbf{Term} & \textbf{Definition}\\
		\midrule
		Cascading Style Sheets & The stylesheet language used to style the 
		project\\
		\hline
		Javascript[JS] & The programming language used to provide interactivity 
		in the 
		project\\
		\hline
		Game Board & A collection of boxes on which the letters will be 
		displayed.\\
		\hline
		Graphical User Interface &  A representation of the program allowing 
		for user 
		interaction.\\
		\hline
		HTML & The markup language used to structure the project\\
		\hline
		Keyboard & The area on the screen where the player selects individual 
		letters.\\
		\hline
		Mocha & One of the testing frameworks used.\\
		\hline
		Not Wordle & The original game on which this project is based.\\
		\hline
		Player & The individual who is playing the game.\\
		\hline
		Player Statistics & A collection of values showing the win/loss rate 
		and number 
		of guesses.\\
		\hline
		Selenium & The UI testing framework used.\\
		\hline
		Software Requirements Specification & A document that describes a 
		software 
		system's performance and functionality.\\
		\hline
		Tester & A person who is testing the program by way of using the 
		program.\\
		\hline
		Typescript & The programming language used in the original Not Wordle 
		game.\\
		
		\bottomrule
	\end{tabularx}
	
\end{table}	

\subsection{Overview of Document}
The document contains a description of the software, the tools and methods used 
to test the software, and the team that will be testing it. It includes 
all the test cases that will be used to verify the software's adherence to the 
all SRS requirements. Also, a comparison to the original Wordle game is 
provided.

\section{Plan}
	
\subsection{Software Description}
Wordle 2.0 is the reimplementation of the game Wordle. It is built using 
JavaScript, CSS, and HTML.
\subsection{Test Team}
All members of Group-8 are responsible for writing and executing tests:
\begin{enumerate}
    \item Richard Fan
    \item Biranugan Pirabaharan
    \item Noel Zacharia
\end{enumerate}
In addition to this, once the product is in the final stages of development, we would seek user feedback by recruiting volunteers.

\subsection{Automated Testing Approach}
As our project is a GUI-based game, automated testing will be a minor aspect of 
our testing framework. We plan on testing via obtaining user feedback and 
verification of different game scenarios. However, we will be writing automated 
unit tests for different components of the game using the Selenium Testing 
Framework. For unit tests concerning internal functions, we will be using Mocha.
\subsection{Testing Tools}
Mocha is a JavaScript testing framework that runs both on Node.js and the 
browser. Mocha will allow for unit tests to be conducted on internal functions 
within the game. The main testing tool we will be using is Selenium. Selenium 
is an 
open-source testing framework for web applications. Selenium allows for 
automated testing of the UI components and thus allows us to verify the 
rendering of page elements. Python will be used to write Selenium test scripts.
\subsection{Testing Schedule}
		
The testing schedule is described in the Gantt chart.

\section{System Test Description}
	
\subsection{Tests for Functional Requirements}

\subsubsection{User Interface Tests}

\begin{enumerate}

\item{FR-UI1\\}

Type: Functional, Manual 
					
Initial State: Empty web browser page.
					
Input: A start game event.
					
Output: An on screen keyboard with a QWERTY layout, containing enter and delete 
keys.
					
How test will be performed: A tester will open the game within a web browser 
and check that an on screen keyboard of QWERTY layout is displayed.
					
\item{FR-UI2\\}

Type: Functional, Manual
					
Initial State: An blank web browser page.
					
Input: A start game event.
					
Output: A game board of with 6 columns and 5 rows.
					
How test will be performed:  A tester will open the game within a web browser 
and check that a game board with 5 rows and 6 columns is displayed.

\item{FR-UI3\\}

Type: Functional, Automated, Dynamic

Initial State: A web browser with the Wordle 2.0 game loaded.

Input: The theme button is pressed.

Output: The theme of the game changes.

How test will be performed: An automated script will click on the theme button 
and 
check if the styling has changed. 

\item{FR-UI4\\}

Type: Functional, Manual, Dynamic

Initial State: A web browser with the Wordle 2.0 game loaded.

Input: The rules button is pressed.

Output: A message box is displayed containing the instructions of the game.

How test will be performed: A tester will click on the instructions button and 
check that a message box containing the instructions is displayed.

\item{FR-UI5\\}

Type: Functional, Manual, Dynamic

Initial State: A web browser with the Wordle 2.0 game loaded.

Input: The statistics button is pressed.

Output: A message box is displayed containing all of the player statistics.

How test will be performed: A tester will click on the statistics button and 
check that a message box containing the player statistics is displayed.

\item{FR-UI6\\}

Type: Functional, Manual, Dynamic

Initial State: A web browser with the Wordle 2.0 game loaded.

Input: An end game event.

Output: A message box is displayed containing the number of games played.

How test will be performed: A tester will do a playthrough of the game. At the 
end they will check that a message box containing the total number of games is 
displayed.

\item{FR-UI7\\}

Type: Functional, Manual, Dynamic

Initial State: A web browser with the Wordle 2.0 game loaded.

Input: An end game event.

Output: A message box is displayed containing the win/loss ratio.

How test will be performed: A tester will do a playthrough of the game. At the 
end, they will check that a message box containing the player's win/loss ratio 
is 
displayed.

\item{FR-UI8\\}

Type: Functional, Manual, Dynamic

Initial State: A web browser with the Wordle 2.0 game loaded.

Input: An end game event.

Output: A message box is displayed containing the current streak of correct 
guesses.

How test will be performed: A tester will do a playthrough of the game. At the 
end, they will check that a message box containing the player's current streak 
of correct guesses is displayed.

\item{FR-UI9\\}

Type: Functional, Manual, Dynamic

Initial State: A web browser with the Wordle 2.0 game loaded.

Input: An end game event.

Output: A message box is displayed containing the best streak of correct 
guesses.

How test will be performed: A tester will do a playthrough of the game. At the 
end they will check that a message box containing the player's best streak 
of correct guesses is displayed.

\item{FR-UI10\\}

Type: Functional, Manual, Dynamic

Initial State: A web browser with the Wordle 2.0 game loaded.

Input: An end game event.

Output: A message box is displayed, which contains the player's guess 
distribution.

How test will be performed: A tester will do a playthrough of the game. At the 
end, they will check if a message box containing the player's guess 
distribution is displayed.

\end{enumerate}
\subsubsection{Gameplay Tests}
\begin{enumerate}
\item{FR-GP1\\}

Type: Functional, Manual

Initial State: The Wordle 2.0 site has been initialized.

Input: Tester clicks the share option.

Output: The clipboard stores a copy of the results.

How test will be performed: A tester will play the game and at the end of the 
game, will click the share option, paste the results into a word document and 
check whether the desired image has been pasted.

\item{FR-GP2\\}

Type: Functional, Manual

Initial State: The Wordle 2.0 site has been initialized.

Input: Tester enters a word.

Output: The gameboard allows the user to enter the word.

How test will be performed: A tester will play the game and enter a 5-letter 
word and hit submit. The tester will observe if the gameboard accepts the word 
or not.

\item{FR-GP3\\}

Type: Functional, Manual

Initial State: The Wordle 2.0 site has been initialized.

Input: Tester enters a word with a letter in the correct spot.

Output: The gameboard highlights the correct letter tile in green.

How test will be performed: A tester will play the game and enter a 5-letter 
word with one letter in the correct spot and will observe whether or not the 
gameboard 
highlights that tile in green.

\item{FR-GP4\\}

Type: Functional, Manual

Initial State: The Wordle 2.0 site has been initialized.

Input: Tester enters a word with a letter in the correct spot.

Output: The gameboard highlights the correct letter tile in green.

How test will be performed: A tester will play the game and enter a 5-letter 
word with one letter in the correct spot and observes whether the gameboard 
highlights that tile in green.

\item{FR-GP5\\}

Type: Functional, Manual

Initial State: The Wordle 2.0 site has been initialized.

Input: Tester enters a word with a letter in the wrong spot.

Output: The gameboard does not highlight the letter tile.

How test will be performed: A tester will play the game and enter a 5-letter 
word with letters in the wrong spot and observes whether the gameboard does not 
highlight that tile.

\item{FR-GP6\\}

Type: Functional, Manual

Initial State: The Wordle 2.0 site has been initialized.

Input: Tester enters an invalid 5-letter word.

Output: The game alerts the user about the invalid word.

How test will be performed: A tester will play the game and enter an invalid 
word. The game should alert the user about this.

\item{FR-GP7\\}

Type: Functional, Manual

Initial State: The Wordle 2.0 site has been initialized.

Input: Tester enters an invalid word with less than 5 letters.

Output: The game alerts the user about the invalid word.

How test will be performed: A tester will play the game and enter an invalid 
word with less than 5 letters. The game should alert the user about this.

\item{FR-GP8\\}

Type: Functional, Manual

Initial State: The Wordle 2.0 site has been initialized.

Input: Tester enters a word with some correct letters.

Output: The game keyboard updates itself to reflect the accuracy of the new 
guess.

How test will be performed: A tester will play the game and enter a word. The 
keyboard will update itself to reflect the letters which were within the word 
and those that were not.

\item{FR-GP9\\}

Type: Functional, Manual

Initial State: The Wordle 2.0 site has been initialized.

Input: Tester enters a word and tries to modify that word after submission.

Output: The game does not allow any modifications after submission.

How test will be performed: A tester will play the game, enter a word and try 
to modify that word after submission, however, the game must ensure that this 
does not happen.

\item{FR-GP10\\}

Type: Functional, Manual

Initial State: The Wordle 2.0 site has been initialized.

Input: Tester sees an option to change the game word length setting.

Output: The game provides the user option to change the word length.

How test will be performed: A tester will see the option to change word length 
in the game and the game allows them to alternate between different modes.

\item{FR-GP11\\}

Type: Functional, Manual

Initial State: The Wordle 2.0 site has been initialized.

Input: Tester clicks the change word-length button.

Output: The game changes the word length setting according to player inputs.

How test will be performed: A tester will change the game length and enter 
words of corresponding lengths to ensure that the game functions as desired in 
all settings.

\item{FR-GP12\\}

Type: Functional, Manual

Initial State: The Wordle 2.0 site has been initialized.

Input: Tester clicks the reset button.

Output: The game resets for another round of play.

How test will be performed: A tester will click the reset button and the game 
should allow them to play another round with a new word.

\end{enumerate}



\subsection{Tests for Nonfunctional Requirements}

\subsubsection{Look and Feel Testing}

\begin{enumerate}

\item{NFR-LF1\\}

Type: Functional, Dynamic, Manual
					
Initial State: Game loaded with no guesses inputed. 
					
Input: \textbf{Tester} plays the game by guessing words.
					
Output: Survey results
					
How test will be performed: The responses of the \textbf{testers} 
to question 2 and 3 of the Usability Survey will determine the 
ease of use and understandability of the game. 

					
\item{NFR-LF2\\}

Type: Functional, Dynamic, Manual
					
Initial State: Game loaded with no guesses made. 
					
Input: \textbf{Tester} plays the game by guessing words.
					
Output: Survey results
					
How test will be performed: \textbf{Testers} will play the 
game and the original \textbf{Wordle 2.0} and answer questions 1 and 4 from the 
Usability Survey.

\end{enumerate}

\subsubsection{Usability and Humanity Testing}

\begin{enumerate}

\item{NRF-UH1\\}

Type: Functional, Dynamic, Manual
					
Initial State: Game loaded with no guesses made. 
					
Input: \textbf{Tester} plays the game by guessing words.
					
Output: Survey results
					
How test will be performed: Children at or above the age of $\hypertarget{min_age}{MIN\_AGE}$ 
will play the game and fill out a survey that evaluates their understanding.

\item{NRF-UH2\\}

Type: Functional, Dynamic, Manual
					
Initial State: Game loaded with no guesses made. 
					
Input: \textbf{Tester} plays the game by guessing words.
					
Output: Survey results
					
How test will be performed: \textbf{Testers} will play the 
game and will fill out a survey that evaluates how easy they 
found the game to be playable with one hand

\item{NRF-UH4\\}

Type: Functional, Dynamic, Manual
					
Initial State: Game loaded with no guesses made. 
					
Input: \textbf{Tester} plays the game by guessing words.
					
Output: Survey Results
					
How test will be performed: The responses of the \textbf{testers} 
to question 5 of the Usability Survey will determine the difficulty of the game. 

\end{enumerate}

\subsubsection{Performance Testing}

\begin{enumerate}

\item{NRF-P1\\}

Type: Functional, Automatic
					
Initial State: Game loaded with no guesses made. 
					
Input: A guess will be inputted.
					
Output: Survey Results
					
How test will be performed: An automated unit test will check the time it takes for the 
page elements to be added to the element tree and be displayed. It must be less than
$\hypertarget{min_time}{MIN\_TIME}$, over multiple executions and varying devices.

\item{NRF-P2\\}

Type: Functional, Automatic
					
Initial State: Game is to be initialized. 
					
Input: The site will be reloaded
					
Output: Survey Results
					
How test will be performed: An automated unit test will reload the page to check the time it takes for the 
page elements to be loaded to the element tree and be displayed. It must be less than
$\hypertarget{min_time}{MIN\_TIME}$, over multiple executions and varying devices.


\end{enumerate}


\subsection{Traceability Between Test Cases and Requirements}

\begin{landscape}

\begin{table}[H]
	% \begin{center} 
		\caption{\textbf{Traceability Matrix for UI Requirements}} 
		\label{trace2}
		\begin{tabularx}{\textwidth}{cc|c|c|c|c|c|c|c|c|c|c|c}
			\cline{3-12}
			& & \multicolumn{10}{c|}{Requirements} \\ \cline{3-12}
			& & FR1 & FR2 & FR3 & FR4 & FR5 & FR6 & FR7 & FR8 & FR9 & 
			FR10  \\ \cline{1-12}
			\multicolumn{1}{|c|}{\multirow{10}{*}{Test Cases} } &
			\multicolumn{1}{|c|} {FR-UI1} &X&&&&&&&&&&\\ \cline{2-12}
			\multicolumn{1}{|c|}{} 	                  &
			\multicolumn{1}{|c|}{FR-UI2} &&X&&&&&&&& \\ \cline{2-12}
			\multicolumn{1}{|c|}{}                        &
			\multicolumn{1}{|c|} {FR-UI3} &&&X&&&&&&& \\ \cline{2-12}
			\multicolumn{1}{|c|}{}                        &
			\multicolumn{1}{|c|} {FR-UI4} &&&&X&&&&&& \\ \cline{2-12}
			\multicolumn{1}{|c|}{}                        &
			\multicolumn{1}{|c|} {FR-UI5} &&&&&X&&&&& \\ \cline{2-12}
			\multicolumn{1}{|c|}{}                        &
			\multicolumn{1}{|c|} {FR-UI6} &&&&&&X&&&& \\ \cline{2-12}
			\multicolumn{1}{|c|}{}                        &
			\multicolumn{1}{|c|} {FR-UI7} &&&&&&&X&&& \\ \cline{2-12}
			\multicolumn{1}{|c|}{}                        &
			\multicolumn{1}{|c|} {FR-UI8} &&&&&&&&X&& \\ \cline{2-12}
			\multicolumn{1}{|c|}{}                        &
			\multicolumn{1}{|c|} {FR-UI9} &&&&&&&&&X& \\ \cline{2-12}
			\multicolumn{1}{|c|}{}                        &
			\multicolumn{1}{|c|} {FR-UI10} &&&&&&&&&&X \\ \cline{1-12}
			
		\end{tabularx}
\end{table}

\begin{table}[H]
	% \begin{center} 
		\caption{\textbf{Traceability Matrix for Gameplay Requirements}} 
		\label{trace2}
		\begin{tabularx}{\textwidth}{cc|c|c|c|c|c|c|c|c|c|c|c|c|c}
			\cline{3-14}
			& & \multicolumn{12}{c|}{Requirements} \\ \cline{3-14}
			& & FR11 & FR12 & FR13 & FR14 & FR15 & FR16 & FR17 & FR18 & FR19 & 
			FR20 & FR21 & FR22  \\ \cline{1-14}
			\multicolumn{1}{|c|}{\multirow{10}{*}{Test Cases} } &
			\multicolumn{1}{|c|} {FR-GP1} &X&&&&&&&&&&&&\\ \cline{2-14}
			\multicolumn{1}{|c|}{} 	                  &
			\multicolumn{1}{|c|}{FR-GP2} &&X&&&&&&&&&& \\ \cline{2-14}
			\multicolumn{1}{|c|}{}                        &
			\multicolumn{1}{|c|} {FR-GP3} &&&X&&&&&&&&& \\ \cline{2-14}
			\multicolumn{1}{|c|}{}                        &
			\multicolumn{1}{|c|} {FR-GP4} &&&&X&&&&&&&& \\ \cline{2-14}
			\multicolumn{1}{|c|}{}                        &
			\multicolumn{1}{|c|} {FR-GP5} &&&&&X&&&&&&& \\ \cline{2-14}
			\multicolumn{1}{|c|}{}                        &
			\multicolumn{1}{|c|} {FR-GP6} &&&&&&X&&&&&& \\ \cline{2-14}
			\multicolumn{1}{|c|}{}                        &
			\multicolumn{1}{|c|} {FR-GP7} &&&&&&&X&&&&& \\ \cline{2-14}
			\multicolumn{1}{|c|}{}                        &
			\multicolumn{1}{|c|} {FR-GP8} &&&&&&&&X&&&& \\ \cline{2-14}
			\multicolumn{1}{|c|}{}                        &
			\multicolumn{1}{|c|} {FR-GP9} &&&&&&&&&X&&& \\ \cline{2-14}
			\multicolumn{1}{|c|}{}                        &
			\multicolumn{1}{|c|} {FR-GP10} &&&&&&&&&&X&& \\ \cline{2-14}
			\multicolumn{1}{|c|}{}                        &
			\multicolumn{1}{|c|} {FR-GP11} &&&&&&&&&&&X& \\ \cline{2-14}
			\multicolumn{1}{|c|}{}                        &
			\multicolumn{1}{|c|} {FR-GP12} &&&&&&&&&&&&X \\ \cline{1-14}
		\end{tabularx}
	\end{table}

\begin{table}[H]
	% \begin{center} 
		\caption{\textbf{Traceability Matrix for Non Functional Requirements}} 
		\label{trace2}
		\begin{tabularx}{\textwidth}{cc|c|c|c|c|c|c|c|c}
			\cline{3-9}
			& & \multicolumn{7}{c|}{Requirements} \\ \cline{3-9}
			& & LF1 & LF2 & UH1 & UH2 & UH4 & P1 & P2  \\ 
			\cline{1-9}
			\multicolumn{1}{|c|}{\multirow{10}{*}{Test Cases} } &
			\multicolumn{1}{|c|} {NFR-LF1} &X&&&&&&\\ \cline{2-9}
			\multicolumn{1}{|c|}{} 	                  &
			\multicolumn{1}{|c|}{NFR-LF2} &&X&&&&& \\ \cline{2-9}
			\multicolumn{1}{|c|}{}                        &
			\multicolumn{1}{|c|} {NFR-UH1} &&&X&&&&\\ \cline{2-9}
			\multicolumn{1}{|c|}{}                        &
			\multicolumn{1}{|c|} {NFR-UH2} &&&&X&&& \\ \cline{2-9}
			\multicolumn{1}{|c|}{}                        &
			\multicolumn{1}{|c|} {NFR-UH4} &&&&&X&& \\ \cline{2-9}
			\multicolumn{1}{|c|}{}                        &
			\multicolumn{1}{|c|} {NFR-P1} &&&&&&X& \\ \cline{2-9}
			\multicolumn{1}{|c|}{}                        &
			\multicolumn{1}{|c|} {NFR-P2} &&&&&&&X \\ \cline{1-9}

		\end{tabularx}
	\end{table}


\end{landscape}




\section{Tests for Proof of Concept}

\subsection{Gameplay Testing}

\begin{enumerate}

\item{POC-GP-1}

Type: Functional, Manual
					
Initial State: Initialization of the Wordle board and keyboard
					
Input: The webpage is reloaded
					
Output: A blank board and default coloured keyboard are displayed.
					
How test will be performed: The tester will load a new game and check that the 
game's guess board and on screen keyboard appear.
					
\item{POC-GP-2}

Type: Functional, Manual
					
Initial State: Game should be initialized
					
Input: \textbf{Tester} submits a word with less than 5 letters.
					
Output: Player is alerted the word is too short.
					
How test will be performed: The \textbf{tester} submits a word with less than 5 letters,
and checks to see if they are alerted in response. 

\item{POC-GP-3}

Type: Functional, Manual
					
Initial State: Game should be initialized
					
Input: \textbf{Tester} submits a fake word.
					
Output: Player is alerted the word is invalid.
					
How test will be performed: The \textbf{tester} submits a gibberish word,
and checks to see if they are alerted in response. 

\item{POC-GP-4}

Type: Functional, Manual
					
Initial State: Game should be initialized
					
Input: \textbf{Tester} guesses various word.
					
Output: The game board and on-screen keyboard updates.
					
How test will be performed: The \textbf{tester} submits various words, 
through both input methods, and checks to see if the game board 
and on-screen keyboard updates with the appropriate colours. 

\item{POC-GP-5}

Type: Functional, Manual
					
Initial State: Game should be initialized
					
Input: The target word is submitted.
					
Output: The game no longer accepts any new guesses.
					
How test will be performed: The test automatically guesses the target word, 
then a \textbf{tester} checks both the physical and on-screen keyboards.  

\item{POC-GP-6}

Type: Functional, Manual
					
Initial State: Game should be initialized
					
Input: The dark mode switch is clicked.
					
Output: The game switches themes.
					
How test will be performed: The \textbf{tester} clicks the dark mode switch,
and checks to see if all the elements switch to their dark mode colours. The tester
should click the switch again to ensure it can go both ways.

\end{enumerate}

	
\section{Comparison to Existing Implementation}	
At the time of writing this document, Wordle and Wordle 2.0 both have mirrored 
core functionality as both provide an interface to input 5-letter words and 
give information on the accuracy of the input letters. Things that remain to be 
implemented are additional menu options for player statistics, dynamic word 
levels and improved graphics.
\section{Unit Testing Plan}
		
\subsection{Unit testing of internal functions}
We will be using Mocha and the built-in assertion library provided by Node.js 
to 
test internal functions. We will write a test file consisting of different 
functions that correspond to tests for each module. We will be using assertion 
statements, which take predetermined parameters and compares the method output 
with the expected output. If the assertion statement evaluates to true, we can 
say that this test has passed successfully and is correct for this case. To 
ensure coverage of the input domain, we will choose extreme/boundary values as 
our predetermined parameters. The Selenium suite will be used also be used to 
test internal functions that can not be tested using Mocha. We will be using a 
page object design pattern for our tests. Selenium will be used to get HTML 
elements from the game and pass inputs to them. Assert statements will then be 
used to compare the system output with the expected output. This may be in the 
form of locating new elements on the game page etc. It is expected that unit 
testing will be able to cover 
$\hypertarget{unit_testing_coverage}{UNIT\_TEST\_COVERAGE}$ percent of the 
code. The rest of the code will be verified using manual testing.
		
\subsection{Unit testing of output files}		
There are no output files created from \textbf{Wordle 2.0}, 
therefore there are no unit tests for this section.	

\newpage

\section{Appendix}

\subsection{Symbolic Parameters}

$\hypertarget{min_age}{MIN\_AGE}$ = 10\\
$\hypertarget{min_time}{MIN\_TIME}$ = 0.25\\
$\hypertarget{unit_testing_coverage}{UNIT\_TEST\_COVERAGE}$ = 40\\
\subsection{Usability Survey Questions?}


Tester feedback is an important component of testing and their experience playing Wordle should be taken into account. The following are some survey questions we could ask:
\begin{enumerate}
    \item How would you compare your experience playing Wordle 2.0 to the official version of Wordle?
    \item On a scale from 1-10, how intuitive were the controls?
    \item On a scale from 1-10, how easy was it to enter inputs within the game?
    \item Would you consider the UI to be visually appealing?
    \item How long did it take you to figure out the instructions on how to play this game?
\end{enumerate}

\end{document}